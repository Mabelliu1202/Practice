\documentclass[cjjs]{ipart}

\RequirePackage{hyperref}

\startlocaldefs
\theoremstyle{plain}
\newtheorem{thm}{Theorem}[section]
\def\SM#1#2{\sum_{#1\in #2}}
\def\FL#1{\left\lfloor #1 \right\rfloor}
\def\FR#1#2{{\frac{#1}{#2}}}
\endlocaldefs

%%  Settings
\pubyear{2025}
\volume{2}
\issue{1}
\firstpage{1}
\lastpage{5}
\arxiv{0000.0000}

\begin{document}

\begin{frontmatter}
\title[This is a book]{This is a book\protect\thanksref{T1}}
\thankstext{T1}{Thanks Professor Yau's support.}

\begin{aug}
    \author{\fnms{Mabel} \snm{Liu}\thanksref{t2}\ead[label=e1]{yassist@math.ntu.edu.tw}},
    \address{No. 120 Da An Road \\
             Taiwan\\
             \printead{e1}}
    \author{\fnms{Leslie} \snm{Liu}\ead[label=e2]{yassist@tims.ntu.edu.tw}},
    \address{ No. 121 Da she Road\\
             Taiwan\\
             \printead{e2}}
    \and
    \author{\fnms{May} \snm{Liu}
            \ead[label=e3]{ml1202tw@yahoo.com.tw}%
            \ead[label=u1,url]{http://www.ntu.edu.tw}}
    \address{No.122 Da Chi Road\\
             Taiwan\\
             \printead{e3}\\
             \printead{u1}}
    \thankstext{t2}{Thanks Professor Yu's support.}
\end{aug}
%
\received{\sday{15} \smonth{4} \syear{2024}}

\begin{abstract} Abstract:
This is a nice book. People love this book very much. 

\end{abstract}


\begin{keyword}
\kwd{AI}
\kwd{GPT}
\kwd{Deepseek}
\end{keyword}


%\tableofcontents
\end{frontmatter}

\section{Introduction}

This book is very interesting about AI description. 

\begin{table*}
\tabcolsep=0pt
\caption{The spherical case ($I_1=0$, $I_2=0$)}\label{sphericcase}
\begin{tabular*}{\textwidth}{@{\extracolsep{\fill}}crrrrc}
\hline
 Equil. Points & \multicolumn{1}{c}{$x$} & \multicolumn{1}{c}{$y$} & \multicolumn{1}{c}{$z$} & \multicolumn{1}{c}{$C$} &
S \\
\hline
$L_1$ & $-$2.485252241 & 0.000000000 & 0.017100631 & 8.230711648 & U \\
$L_2$ &    0.000000000 & 0.000000000 & 3.068883732 & 0.000000000 & S \\
$L_3$ &    0.009869059 & 0.000000000 & 4.756386544 & $-$0.000057922 & U \\
$L_4$ &    0.210589855 & 0.000000000 & $-$0.007021459 & 9.440510897 & U \\
$L_5$ &    0.455926604 & 0.000000000 & $-$0.212446624 & 7.586126667 & U \\
$L_6$ &    0.667031314 & 0.000000000 & 0.529879957 & 3.497660052 & U \\
$L_7$ &    2.164386674 & 0.000000000 & $-$0.169308438 & 6.866562449 & U \\
$L_8$ &    0.560414471 & 0.421735658 & $-$0.093667445 & 9.241525367 & U \\
$L_9$ &    0.560414471 & $-$0.421735658 & $-$0.093667445 & 9.241525367 & U\\
$L_{10}$ & 1.472523232 & 1.393484549 & $-$0.083801333 & 6.733436505 & U \\
$L_{11}$ & 1.472523232 & $-$1.393484549 & $-$0.083801333 & 6.733436505 & U
\\ \hline
\end{tabular*}
\end{table*}

\section{Definition}
This is a BOOK AI1.0 and AI2.0. 

Book\footnote{This is a sample of book.}
and paper\footnote{This is a sample of paper}.

\section{Lemmas & Theorems}


\begin{itemize}
\item  This is a AI1.0 book. 
\item  This is a AI2.0 book. 


\[  x' + y^{2} = z_{i}^{2}.\] This is a AI1 book. 
\\Example of a theorem:

\begin{thm}
It is interesting about AI problems. 
\end{thm}

\begin{proof}
Obvious result. 
\end{proof}

\section{Proofs}

and also in Table~\ref{parset}.

\begin{table}
\centering
\caption{Parameter sets used by Bajpai and Reu\ss are not detailed sufficiently to permit a similar analysis}
\label{parset}
\begin{tabular}{lrll}
\hline
\multicolumn{2}{@{}l}{Parameter} & Set 1 & Set 2\\
\hline
$\mu_{x}$           & [h$^{-1}$]  & 0.092       & 0.11          \\
$K_{x}$             & [g/g DM]     & 0.15        & 0.006         \\
$\mu_{p}$           & [g/g DM h]  & 0.005       & 0.004         \\
$K_{p}$             & [g/L]        & 0.0002      & 0.0001        \\
$K_{i}$             & [g/L]        & 0.1         & 0.1           \\
$Y_{x/s}$           & [g DM/g]     & 0.45        & 0.47          \\
$Y_{p/s}$           & [g/g]        & 0.9         & 1.2           \\
$k_{h}$             & [h$^{-1}$]  & 0.04        & 0.01          \\
$m_{s}$             & [g/g DM h]  & 0.014       & 0.029         \\
\hline
\end{tabular}
\end{table}

\begin{figure}
\centering
\fbox{\makebox[5cm][c]{Picture}\rule[-1.5cm]{0pt}{3cm}}
%\includegraphics{}\input{Test 3}
\caption{Pathway of the penicillin G biosynthesis.}\label{penG}
\end{figure}


\section{Conclusion}
This is a AI1.0 & AI2.0. 
\subsection{This is a book AI3.0}
This is a book AI3.0.

Two equations:
\begin{equation}
    C_{s}  =  K_{M} \frac{\mu/\mu_{x}}{1-\mu/\mu_{x}} \label{ccs}
\end{equation}
and
\begin{equation}
    G = \frac{P_{\rm opt} - P_{\rm ref}}{P_{\rm ref}} \mbox{\ }100 \mbox{\ }(\%).
\end{equation}

Two equation arrays:
\begin{align}
  \frac{dS}{dt} & =  - \sigma X + s_{F} F\\
  \frac{dX}{dt} & =    \mu    X\\
  \frac{dP}{dt} & =    \pi    X - k_{h} P\\
  \frac{dV}{dt} & =    F
\end{align}
and
\begin{align}
 \mu_{\rm substr} & =  \mu_{x} \frac{C_{s}}{K_{x}C_{x}+C_{s}}  \\
 \mu              & =  \mu_{\rm substr} - Y_{x/s}(1-H(C_{s}))(m_{s}+\pi /Y_{p/s}) \\
 \sigma           & =  \mu_{\rm substr}/Y_{x/s}+ H(C_{s}) (m_{s}+ \pi /Y_{p/s})
\end{align}

Long equation:
\begin{align}
\SM u{C^+}\FL{\FR{w'(u)}m}&\le \FL{\SM u{C^+} \FR{w'(u)}m}\nonumber\\ &\le
\FL{\FR{r(v)+\SM u{C^+} w(u)}m} = \FL{\FR{w(v)-\SM u{C^-} w(u)}m}\nonumber\\
&\le \FL{\FR{w(v)}m}-\SM u{C^-}\FL{\FR{w(u)}m}
= s(v)+\SM u{C^+}\FL{\FR{w(u)}m}
\end{align}
and
\begin{align}
\SM u{C^-}\FL{\FR{w(u)}m}&\le \FL{\SM u{C^-} \FR{w(u)}m}\nonumber\\ & \le
\FL{\FR{r'(v)+\SM u{C^-} w'(u)}m} = \FL{\FR{w'(v)-\SM u{C^+} w'(u)}m}\nonumber\\
&\le \FL{\FR{w'(v)}m}-\SM u{C^+}\FL{\FR{w'(u)}m}
= s'(v)+\SM u{C^-}\FL{\FR{w'(u)}m}.
\end{align}
This time we have
\begin{align*}
f(S)-f(T) =    {}  &  D_k^T(1+C_{\geq k+1}^T)(1 + C) + C_k^T(1+D_{\geq k+1}^T)(1 + D) \\
        {}   &  - C_k^T(1+C_{\geq k+1}^T)(1 + C) - D_k^T(1+D_{\geq k+1}^T)(1 + D)   \\
=     {} &   (D_k^T-C_k^T)[(1+C_{\geq k+1})(1+C)-(1+D_{\geq
k+1}^T)(1+D)]>0.
\end{align*}

\appendix

\section{Appendix section}\label{app}

An interesting books about AI. 

\subsection{Appendix subsection}


\mathcal{P}=\bigl(j_{k,1},j_{k,2},\dots,j_{k,m(k)}\bigr). \label{path}
\end{equation}

Sample of cross-reference to the formula \ref{path} in Appendix~\ref{app}.

\section*{Acknowledgements}
An interesting books about AI. 

\begin{thebibliography}{9}

\bibitem{r1} C. A. Athanasiadis, On a refinement of the generalized
Catalan numbers for Weyl groups. \textit{Trans. AMS} \textbf{357} (2005), 179--196.
\MR{2098091}

\bibitem{r2} D. Bessis, The dual braid monoid,
\textit{Ann. Sci. \'Ecole Norm. Sup.} S\'er. IV \textbf{36} (2003), 647--683.
\MR{2032983}
\bibitem{r3}L. Lamport, \textit{LaTeX: A Document Preparation System}, 
Addison-Wesley, 2nd Edition, 1994.
\end{thebibliography}

\end{document}
